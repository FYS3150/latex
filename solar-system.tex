\documentclass[a4paper]{article}
\usepackage[english]{babel}
\usepackage[utf8]{inputenc}
\usepackage{amsmath}
\usepackage{graphicx}
\usepackage[colorinlistoftodos]{todonotes}
\usepackage [utf8]{inputenc}
\usepackage{graphicx}
\usepackage{subfigure}
\usepackage{textcomp}
\usepackage{caption}
\usepackage{vmargin}
\usepackage[english]{babel}
\usepackage{amssymb}
\usepackage{mathrsfs}
\usepackage{enumerate}
 \usepackage{booktabs}
  \usepackage{tabularx, array}
  \usepackage{dcolumn}
\usepackage[T1]{fontenc}
\usepackage[labelformat=empty]{caption}
\usepackage{amsmath}
\usepackage{listings}
\usepackage{hyperref}
\usepackage{array}
\usepackage{cancel}
\usepackage{color}
\usepackage{colortbl}
\allowdisplaybreaks

\lstnewenvironment{codice_arduino}[1][]
{\lstset{basicstyle=\small\ttfamily, columns=fullflexible,
keywordstyle=\color{red}\bfseries, commentstyle=\color{blue},
language=C++, 
numbers=left, stepnumber=1, numbersep=4pt, frame=shadowbox,#1}}{}


\title{\textbf{Simulation of the solar system}}

\author{Massimo Giordano, Benjamin Haas}

\date{\today}



\begin{document}
\maketitle
\begin{center}
\href{https://github.com/massimogiordano/repository/tree/master/restore_SOL/}{Link to the repository - code of the program\\jasgnön}
\end{center}

\begin{abstract}
The following article describes a way simulate the solar system considering the sun, the earth, and earth's moon. 
\end{abstract}
\tableofcontents
\pagebreak

\section{Introduction}

\section{Theoretical Concept}
 
\section{Method}

\section{Tuned Program}
\section{Energy states and probability functions}\label{2electrons}

\pagebreak
\section{Analytic Solution}

\pagebreak
\section{Conculsions}

\section{References}
\begin{itemize}
\item M. Hjorth-Jensen, 
Computational Physics, University of Oslo (2013).
\item M. Hjorth-Jensen, 
Project2, University of Oslo (2013).
\item www.cplusplus.com
\end{itemize}



\end{document}